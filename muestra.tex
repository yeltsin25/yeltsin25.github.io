% Options for packages loaded elsewhere
\PassOptionsToPackage{unicode}{hyperref}
\PassOptionsToPackage{hyphens}{url}
%
\documentclass[
]{article}
\usepackage{amsmath,amssymb}
\usepackage{lmodern}
\usepackage{iftex}
\ifPDFTeX
  \usepackage[T1]{fontenc}
  \usepackage[utf8]{inputenc}
  \usepackage{textcomp} % provide euro and other symbols
\else % if luatex or xetex
  \usepackage{unicode-math}
  \defaultfontfeatures{Scale=MatchLowercase}
  \defaultfontfeatures[\rmfamily]{Ligatures=TeX,Scale=1}
\fi
% Use upquote if available, for straight quotes in verbatim environments
\IfFileExists{upquote.sty}{\usepackage{upquote}}{}
\IfFileExists{microtype.sty}{% use microtype if available
  \usepackage[]{microtype}
  \UseMicrotypeSet[protrusion]{basicmath} % disable protrusion for tt fonts
}{}
\makeatletter
\@ifundefined{KOMAClassName}{% if non-KOMA class
  \IfFileExists{parskip.sty}{%
    \usepackage{parskip}
  }{% else
    \setlength{\parindent}{0pt}
    \setlength{\parskip}{6pt plus 2pt minus 1pt}}
}{% if KOMA class
  \KOMAoptions{parskip=half}}
\makeatother
\usepackage{xcolor}
\IfFileExists{xurl.sty}{\usepackage{xurl}}{} % add URL line breaks if available
\IfFileExists{bookmark.sty}{\usepackage{bookmark}}{\usepackage{hyperref}}
\hypersetup{
  pdftitle={DISTRIBUCIONES DISCRETA},
  hidelinks,
  pdfcreator={LaTeX via pandoc}}
\urlstyle{same} % disable monospaced font for URLs
\usepackage[margin=1in]{geometry}
\usepackage{color}
\usepackage{fancyvrb}
\newcommand{\VerbBar}{|}
\newcommand{\VERB}{\Verb[commandchars=\\\{\}]}
\DefineVerbatimEnvironment{Highlighting}{Verbatim}{commandchars=\\\{\}}
% Add ',fontsize=\small' for more characters per line
\usepackage{framed}
\definecolor{shadecolor}{RGB}{248,248,248}
\newenvironment{Shaded}{\begin{snugshade}}{\end{snugshade}}
\newcommand{\AlertTok}[1]{\textcolor[rgb]{0.94,0.16,0.16}{#1}}
\newcommand{\AnnotationTok}[1]{\textcolor[rgb]{0.56,0.35,0.01}{\textbf{\textit{#1}}}}
\newcommand{\AttributeTok}[1]{\textcolor[rgb]{0.77,0.63,0.00}{#1}}
\newcommand{\BaseNTok}[1]{\textcolor[rgb]{0.00,0.00,0.81}{#1}}
\newcommand{\BuiltInTok}[1]{#1}
\newcommand{\CharTok}[1]{\textcolor[rgb]{0.31,0.60,0.02}{#1}}
\newcommand{\CommentTok}[1]{\textcolor[rgb]{0.56,0.35,0.01}{\textit{#1}}}
\newcommand{\CommentVarTok}[1]{\textcolor[rgb]{0.56,0.35,0.01}{\textbf{\textit{#1}}}}
\newcommand{\ConstantTok}[1]{\textcolor[rgb]{0.00,0.00,0.00}{#1}}
\newcommand{\ControlFlowTok}[1]{\textcolor[rgb]{0.13,0.29,0.53}{\textbf{#1}}}
\newcommand{\DataTypeTok}[1]{\textcolor[rgb]{0.13,0.29,0.53}{#1}}
\newcommand{\DecValTok}[1]{\textcolor[rgb]{0.00,0.00,0.81}{#1}}
\newcommand{\DocumentationTok}[1]{\textcolor[rgb]{0.56,0.35,0.01}{\textbf{\textit{#1}}}}
\newcommand{\ErrorTok}[1]{\textcolor[rgb]{0.64,0.00,0.00}{\textbf{#1}}}
\newcommand{\ExtensionTok}[1]{#1}
\newcommand{\FloatTok}[1]{\textcolor[rgb]{0.00,0.00,0.81}{#1}}
\newcommand{\FunctionTok}[1]{\textcolor[rgb]{0.00,0.00,0.00}{#1}}
\newcommand{\ImportTok}[1]{#1}
\newcommand{\InformationTok}[1]{\textcolor[rgb]{0.56,0.35,0.01}{\textbf{\textit{#1}}}}
\newcommand{\KeywordTok}[1]{\textcolor[rgb]{0.13,0.29,0.53}{\textbf{#1}}}
\newcommand{\NormalTok}[1]{#1}
\newcommand{\OperatorTok}[1]{\textcolor[rgb]{0.81,0.36,0.00}{\textbf{#1}}}
\newcommand{\OtherTok}[1]{\textcolor[rgb]{0.56,0.35,0.01}{#1}}
\newcommand{\PreprocessorTok}[1]{\textcolor[rgb]{0.56,0.35,0.01}{\textit{#1}}}
\newcommand{\RegionMarkerTok}[1]{#1}
\newcommand{\SpecialCharTok}[1]{\textcolor[rgb]{0.00,0.00,0.00}{#1}}
\newcommand{\SpecialStringTok}[1]{\textcolor[rgb]{0.31,0.60,0.02}{#1}}
\newcommand{\StringTok}[1]{\textcolor[rgb]{0.31,0.60,0.02}{#1}}
\newcommand{\VariableTok}[1]{\textcolor[rgb]{0.00,0.00,0.00}{#1}}
\newcommand{\VerbatimStringTok}[1]{\textcolor[rgb]{0.31,0.60,0.02}{#1}}
\newcommand{\WarningTok}[1]{\textcolor[rgb]{0.56,0.35,0.01}{\textbf{\textit{#1}}}}
\usepackage{graphicx}
\makeatletter
\def\maxwidth{\ifdim\Gin@nat@width>\linewidth\linewidth\else\Gin@nat@width\fi}
\def\maxheight{\ifdim\Gin@nat@height>\textheight\textheight\else\Gin@nat@height\fi}
\makeatother
% Scale images if necessary, so that they will not overflow the page
% margins by default, and it is still possible to overwrite the defaults
% using explicit options in \includegraphics[width, height, ...]{}
\setkeys{Gin}{width=\maxwidth,height=\maxheight,keepaspectratio}
% Set default figure placement to htbp
\makeatletter
\def\fps@figure{htbp}
\makeatother
\setlength{\emergencystretch}{3em} % prevent overfull lines
\providecommand{\tightlist}{%
  \setlength{\itemsep}{0pt}\setlength{\parskip}{0pt}}
\setcounter{secnumdepth}{-\maxdimen} % remove section numbering
\ifLuaTeX
  \usepackage{selnolig}  % disable illegal ligatures
\fi

\title{DISTRIBUCIONES DISCRETA}
\author{}
\date{\vspace{-2.5em}}

\begin{document}
\maketitle

\hypertarget{distribuciuxf3n-binomial}{%
\section{1. Distribución Binomial}\label{distribuciuxf3n-binomial}}

Es aquella función que representa cuál es la probabilidad de obtener
\textbf{x} éxitos en \textbf{n} pruebas de Bernoulli independientes,
cuya probabilidad de éxito es de \textbf{p}. \pagebreak la probabilidad
de que se obtengan \textbf{k} éxitos está dada por la función de
probabilidad \textbf{f}

\[f(k) \;=\; P(X=k)  \;=\; {n\choose k} p^k\, (1-p)^{n-k}, \qquad k=0,1,2, \ldots, n\]
\textbf{\emph{Con Esperanza y varianza de}}
\[E(X)= np, \qquad V(X)= np(1-p)\]

\hypertarget{funciuxf3n-de-probabilidad-en-r.}{%
\paragraph{Función de probabilidad en
R.}\label{funciuxf3n-de-probabilidad-en-r.}}

\begin{Shaded}
\begin{Highlighting}[]
\NormalTok{dbinom }\CommentTok{\#Función de masa de probabilidad Binomial (Función de probabilidad)}
\end{Highlighting}
\end{Shaded}

\begin{verbatim}
## function (x, size, prob, log = FALSE) 
## .Call(C_dbinom, x, size, prob, log)
## <bytecode: 0x000002772e41e9b0>
## <environment: namespace:stats>
\end{verbatim}

\begin{Shaded}
\begin{Highlighting}[]
\NormalTok{pbinom }\CommentTok{\#Distribución binomial (Función de distribución acumulada)}
\end{Highlighting}
\end{Shaded}

\begin{verbatim}
## function (q, size, prob, lower.tail = TRUE, log.p = FALSE) 
## .Call(C_pbinom, q, size, prob, lower.tail, log.p)
## <bytecode: 0x00000277270a76a8>
## <environment: namespace:stats>
\end{verbatim}

\begin{Shaded}
\begin{Highlighting}[]
\NormalTok{qbinom }\CommentTok{\#    Función cuantil binomial}
\end{Highlighting}
\end{Shaded}

\begin{verbatim}
## function (p, size, prob, lower.tail = TRUE, log.p = FALSE) 
## .Call(C_qbinom, p, size, prob, lower.tail, log.p)
## <bytecode: 0x000002772e2dc260>
## <environment: namespace:stats>
\end{verbatim}

\begin{Shaded}
\begin{Highlighting}[]
\NormalTok{rbinom  }\CommentTok{\#Generación de números pseudoaleatorios binomiales}
\end{Highlighting}
\end{Shaded}

\begin{verbatim}
## function (n, size, prob) 
## .Call(C_rbinom, n, size, prob)
## <bytecode: 0x000002772e33a4e8>
## <environment: namespace:stats>
\end{verbatim}

\hypertarget{ejercicio-de-ejemplo.}{%
\subsection{Ejercicio de ejemplo.}\label{ejercicio-de-ejemplo.}}

\textbf{\emph{la probabilidad de que el éxito ocurra menos de 3 veces si
el número de ensayos es 10 y la probabilidad de éxito por ensayo es 0.3
es}}

\begin{Shaded}
\begin{Highlighting}[]
\FunctionTok{pbinom}\NormalTok{(}\DecValTok{3}\NormalTok{, }\AttributeTok{size =} \DecValTok{10}\NormalTok{, }\AttributeTok{prob =} \FloatTok{0.3}\NormalTok{)}
\end{Highlighting}
\end{Shaded}

\begin{verbatim}
## [1] 0.6496107
\end{verbatim}

\hypertarget{distribuciuxf3n-hipergeomuxe9trica}{%
\section{2. Distribución
hipergeométrica}\label{distribuciuxf3n-hipergeomuxe9trica}}

Recuérdese que si se selecciona una muestra aleatoria de \textbf{n}
consumidores de una población de \textbf{N} consumidores, el número
\textbf{x} de usuarios que favorecen un producto específico tendría una
distribución binomial cuando el tamaño muestra n es pequeño respecto al
número de \textbf{N} de consumidores en la población, el número
\textbf{x} a favor del producto tiene una distribución de probabilidad
hipergeométrica, \emph{cuya fórmula es:} \pagebreak La correspondiente
distribución de X se conoce con el nombre de distribución
hipergeométrica con parámetros \textbf{n}, \textbf{M} y \textbf{N}.
\[ f(k) \;=\; P(X=k)\;= \;  \frac{{M\choose k}\,{N-M\choose n-k}}{{N\choose n}}, \qquad  \text{donde}\quad k=0,1,2, \ldots, n \quad \text{y}\quad n\leq N \]
- \textbf{n:} Número de elementos en el muestra. - \textbf{M:} Número de
elementos que tienen una característica especifica, por ejemplo el
número de personas a favor un producto particular - \textbf{N:} Número
de elementos en la población. - \textbf{K:} La probabilidad de elegir de
manera exacta k éxitos

\hypertarget{r-distribuciuxf3n-hipergeomuxe9trica.}{%
\subsubsection{R: Distribución
Hipergeométrica.}\label{r-distribuciuxf3n-hipergeomuxe9trica.}}

\begin{Shaded}
\begin{Highlighting}[]
\FunctionTok{dhyper}\NormalTok{(x, m, n, k, }\AttributeTok{log =}\NormalTok{ F) }\CommentTok{\#Devuelve resultados de la función de densidad.}
\FunctionTok{phyper}\NormalTok{(q, m, n, k, }\AttributeTok{lower.tail =}\NormalTok{ T, }\AttributeTok{log.p =}\NormalTok{ F)}\CommentTok{\#Devuelve resultados de la función de distribución acumulada.}
\FunctionTok{qhyper}\NormalTok{(p, m, n, k, }\AttributeTok{lower.tail =}\NormalTok{ T, }\AttributeTok{log.p =}\NormalTok{ F)}\CommentTok{\#Devuelve resultados de los cuantiles de la Hipergeométrica.}
\FunctionTok{rhyper}\NormalTok{(nn, m, n, k)}\CommentTok{\#Devuelve un vector de valores de la Hipergeométrica aleatorios\#}
\end{Highlighting}
\end{Shaded}

\textbf{donde:} \textbf{\emph{Los argumentos que podemos pasar a las
funciones expuestas en la anterior tabla, son:}}

\begin{itemize}
\item
  \textbf{x}, \textbf{q}: Vector de cuantiles. Corresponde al número de
  particulares en la muestra.
\item
  \textbf{m:} Selección aleatoria particular
\item
  \textbf{n:} El número total de la población menos la selección
  aleatoria particular. \textbf{n = N - m}
\item
  \textbf{n:} El número de la selección a evaluar.
\item
  \textbf{prob:} Probabilidad.
\item
  \textbf{nn:} Número de observaciones.
\item
  \textbf{log}, \textbf{log.p:} Parámetro booleano, si es \textbf{TRUE},
  las probabilidades p son devueltas como \textbf{log (p)}.
\item
  \textbf{lower.tail}: Parámetro booleano, si es TRUE (por defecto), las
  probabilidades son \textbf{P{[}X ≤ x{]}}, de lo contrario, \textbf{P
  {[}X \textgreater{} x{]}}
\end{itemize}

\hypertarget{ejercicio-de-ejemplo}{%
\subsection{Ejercicio de Ejemplo}\label{ejercicio-de-ejemplo}}

**De un grupo de 20 ingenieros con doctorado, se eligen 10
aleatoriamente con el fin de contratarlos.

¿Cuál es la probabilidad de que entre los 10 seleccionados, estén los 5
mejores del grupo de 20?**

\hypertarget{soluciuxf3n}{%
\paragraph{\texorpdfstring{\textbf{Solución}}{Solución}}\label{soluciuxf3n}}

\begin{itemize}
\tightlist
\item
  \textbf{N = 20}. Número total de ingenieros.
\item
  \textbf{n = 10.}* Muestra aleatoria de la población total de
  ingenieros \textbf{(20 ingenieros).}
\item
  \textbf{r = 5.} Conjunto de 5 ingenieros estén los 5 mejores.
\end{itemize}

\begin{Shaded}
\begin{Highlighting}[]
\FunctionTok{dhyper}\NormalTok{(}\DecValTok{5}\NormalTok{,}\DecValTok{10}\NormalTok{,}\DecValTok{20{-}10}\NormalTok{,}\DecValTok{5}\NormalTok{)}
\end{Highlighting}
\end{Shaded}

\begin{verbatim}
## [1] 0.01625387
\end{verbatim}

\hypertarget{distribucion-de-poisson.}{%
\section{3.Distribuci´on de Poisson.}\label{distribucion-de-poisson.}}

se usa para modelar el número de eventos que ocurren en un proceso de
Poisson. Sea \textbf{X ∼ P(λ)X∼P(λ)}, esto es, una variable aleatoria
con distribución de Poisson donde el número medio de eventos que ocurren
en un determinado intervalo es \textbf{λ}.

\textbf{\emph{En relación a esta distribución, R tiene 4 funciones:}}

\begin{itemize}
\item
  \textbf{rpois:} genera valores aleatorios acorde a los parámetros
  indicados.
\item
  \textbf{dpois:} calcula la probabilidad puntual para un valor
  específico.
\item
  \textbf{ppois:} proporciona la probabilidad acumulada para un cuantil
  específico.
\item
  \textbf{qpois:} proporciona el cuantil para una probabilidad
  específica
\end{itemize}

\hypertarget{ejercicio-de-ejemplo-1}{%
\subsection{Ejercicio de Ejemplo}\label{ejercicio-de-ejemplo-1}}

\textbf{El número medio de enfermos recibidos cada 10 minutos en un
centro sanitario entre las 10 horas y las 15 horas es 1.8. Suponiendo
que dicho número de enfermos sigue una distribución de Poisson.}
\textbf{\emph{Calcular la probabilidad de que entre las 12 horas y las
12 horas y 10 minutos haya: Exactamente 2 enfermos}}

\begin{Shaded}
\begin{Highlighting}[]
\FunctionTok{dpois}\NormalTok{(}\DecValTok{2}\NormalTok{, }\FloatTok{1.8}\NormalTok{)}
\end{Highlighting}
\end{Shaded}

\begin{verbatim}
## [1] 0.2677842
\end{verbatim}

\end{document}
